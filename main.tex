
\documentclass{article}
\usepackage{digra}

\title{Your Title goes Here: It May Carry \\
Over onto a Second Line}
\author{First Author}
\affil{Institutional Affiliation \\
Address line 1 \\
Address line 2 \\
telephone \\
firstauthor@institution.com }

\author{Second Author, Third Author}
\affil{Institutional Affiliation \\
Address line 1 \\
Address line 2 \\
telephone \\
secondauthor@institution.com, thirdauthor@institution.com}
\date{\vspace{-5ex}}

\begin{document}
\pagenumbering{gobble} 
\newpage
\pagenumbering{arabic}  
   \maketitle
    
    
   \section*{ABSTRACT}
   
Place your abstract here. Your abstract goes here. In this paper we describe the formatting
requirements for DiGRA Conference Proceedings, and offer recommendations on writing
for the worldwide DiGRA readership.
\section*{Keywords}
keyword, keyword, keyword, keyword

\section*{INTRODUCTION}

Place your text here. Your text goes here. This format is to be used for submissions that
are published in the electronic conference proceedings for DiGRA conferences1
. The
same format will be used for conference articles uploaded to the DiGRA library.

In essence, you should format your paper exactly like this document. The easiest way to
do this is simply to download this template from the conference web site, and replace the
content with your own material. The template file contains specially formatted styles (e.g.
Normal, HEADING levels 1 – 3, Reference and Index) that will reduce your work in
formatting your submission.
\section*{PAGE SIZE AND COLUMNS}
The DiGRA proceedings are formatted for a US Letter format, single column page. The
size is chosen to allow for on-screen readability. On each page your material (not
including the page number) should fit within a rectangle of 14 x 22 cm (5.5 x 8.75 in.),
centered on a US letter page, beginning 2.54 cm (1 in.) from the top of the page. On an
A4 page, use a text area of the same dimensions, again centered. Right margins should be
justified, not ragged. Beware that Word can change these dimensions in unexpected
ways.
Although this template has been developed for Word, you can use any word processor or
text system to prepare your submission. Your article should be submitted as a PDF file.
\section*{SECTIONS}
Section headings should be typeset in 12-point Arial bold. Section headers should be set
in all capitals. Leave space corresponding to approximately one line above each heading.
\subsection*{Subsections}
Headings of subsections should be in 12-point Arial bold with initial letters capitalized.
(Note: For sub-sections, a word like the or of is not capitalized unless it is the first word
of the heading.)
\subsubsection*{Sub-subsections}
Headings of sub-subsections should be in 12-point Arial italic, with the first letter
capitalized. Do not use more than three levels of headings.
\section*{BODY TEXT}
For body text, please use a 11-point Times Roman font or, if this is unavailable, another
proportional font with serifs, as close as possible in appearance to Times Roman 12-
point. The Press 10-point font available to users of Script is a good substitute for Times
Roman. If Times Roman is not available, try the font named Computer Modern Roman.
Please use sans-serif or non-proportional fonts only for special purposes, such as
headings or source code text. Note that Body Bulletpointlist, Reference are pre-defined
styles in this template.
\section*{TITLE AND AUTHORS}
Your paper’s title, authors and affiliations should run across the full width of the page
and centered. The title should be in Arial 18-point bold; use Helvetica if Arial is not
available. Authors’ names should be in Times Roman 14-point bold, and affiliations in
Times Roman 12-point (note that AuthorName, AuthorAddress and FirstAuthorEmail as
well as LastAuthorEmail are defined Styles in this template file).

To position names and addresses; use a centered tab stop to center all name and address
text on the page. Place author names and addresses below each other. For multiple
authors with the same affiliation, place the names at the same line with the address below
both. For more than three authors, you may have to place some address information in a
footnote, or in a named section at the end of your paper. Please use full international
addresses and telephone dialing prefixes.

As the template primarily has been developed for online readability, it is allowed to
include active links to web pages and email addresses.
\section*{Abstract and Keywords}
Every submission should begin with an abstract of up to 150 words, followed by a set of
keywords. The abstract should be a concise statement of the problem, approach and
conclusions of the work described. It should clearly state the paper's contribution to the
DiGRA community
\end{document}