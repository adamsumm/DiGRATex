% !BIB TS-program = biber
\documentclass[11pt]{article}
\usepackage{digra}
 
\usepackage[authordate,minnames=1,maxnames=2,maxbibnames=10,minbibnames=7]{biblatex-chicago}
\AtEveryBibitem{\clearfield{doi}\clearfield{urlyear}\clearfield{urlmonth}\clearfield{urlday}}
\DeclareFieldFormat[game]{title}{\mkbibemph{#1}}
\DefineBibliographyStrings{english}{%
  references = {},
}
\let\originalcite\cite
\renewcommand*{\cite}[1]{(\originalcite{#1})}

\addbibresource{bibliography.bib}
\title{\addvspace{-2\baselineskip}Your Title goes Here: It May Carry \\
Over onto a Second Line}
\author{First Author}
\affil{Institutional Affiliation \\
Address line 1 \\
Address line 2 \\
telephone \\
firstauthor@institution.com }

\author{Second Author, Third Author}
\affil{Institutional Affiliation \\
Address line 1 \\
Address line 2 \\
telephone \\
secondauthor@institution.com, thirdauthor@institution.com}
\date{\vspace{-60pt}}

\begin{document}
\pagenumbering{gobble} 
\newpage
\pagenumbering{arabic}  
\addvspace{-1\baselineskip}
   \maketitle
   \addvspace{-1\baselineskip}
    \copyrightnotice
    
   \section*{ABSTRACT}
   
Place your abstract here. Your abstract goes here. In this paper we describe the formatting
requirements for DiGRA Conference Proceedings, and offer recommendations on writing
for the worldwide DiGRA readership.
\section*{Keywords}
keyword, keyword, keyword, keyword

\section*{INTRODUCTION}

Place your text here. Your text goes here. This format is to be used for submissions that
are published in the electronic conference proceedings for DiGRA conferences\endnote{ The format was developed for DiGRA 2011, modified slightly for DiGRA Nordic 2012 and FDG/DiGRA 2016, and may change for future conferences. A slightly different format may apply for the forthcoming DiGRA journal.}
. The
same format will be used for conference articles uploaded to the DiGRA library.

In essence, you should format your paper exactly like this document. The easiest way to
do this is simply to download this template from the conference web site, and replace the
content with your own material.

\section*{PAGE SIZE AND COLUMNS}
The DiGRA proceedings are formatted for a US Letter format, single column page. The
size is chosen to allow for on-screen readability. On each page your material (not
including the page number) should fit within a rectangle of 14 x 22 cm (5.5 x 8.75 in.),
centered on a US letter page, beginning 2.54 cm (1 in.) from the top of the page. On an
A4 page, use a text area of the same dimensions, again centered. Right margins should be
justified, not ragged. Beware that Word can change these dimensions in unexpected
ways.

\section*{SECTIONS}
Section headings should be typeset in 12-point Arial bold. Section headers should be set
in all capitals. Leave space corresponding to approximately one line above each heading.

In LaTeX to avoid numbering of sections you have to use \textbackslash section*\{\} instead of \textbackslash section\{\}
\subsection*{Subsections}
Headings of subsections should be in 12-point Arial bold with initial letters capitalized.
(Note: For sub-sections, a word like the or of is not capitalized unless it is the first word
of the heading.)

In LaTeX to avoid numbering of sections you have to use \textbackslash subsection*\{\} instead of \textbackslash subsection\{\}.
\subsubsection*{Sub-subsections}
Headings of sub-subsections should be in 12-point Arial italic, with the first letter
capitalized. Do not use more than three levels of headings. 

In LaTeX to avoid numbering of sections you have to use \textbackslash subsubsection*\{\} instead of \textbackslash subsubsection\{\}
\section*{BODY TEXT}
For body text, please use a 11-point Times Roman font or, if this is unavailable, another
proportional font with serifs, as close as possible in appearance to Times Roman 11-
point. The Press 10-point font available to users of Script is a good substitute for Times
Roman. If Times Roman is not available, try the font named Computer Modern Roman.
Please use sans-serif or non-proportional fonts only for special purposes, such as
headings or source code text. 

\section*{TITLE AND AUTHORS}
Your paper's title, authors and affiliations should run across the full width of the page
and centered. The title should be in Arial 18-point bold; use Helvetica if Arial is not
available. Authors' names should be in Times Roman 14-point bold, and affiliations in
Times Roman 12-point (note that AuthorName, AuthorAddress and FirstAuthorEmail as
well as LastAuthorEmail are defined Styles in this template file).

Place author names and addresses below each other. For multiple
authors with the same affiliation, place the names at the same line with the address below
both. For more than three authors, you may have to place some address information in a
footnote, or in a named section at the end of your paper. Please use full international
addresses and telephone dialing prefixes.

As the template primarily has been developed for online readability, it is allowed to
include active links to web pages and email addresses.
\section*{Abstract and Keywords}
Every submission should begin with an abstract of up to 150 words, followed by a set of
keywords. The abstract should be a concise statement of the problem, approach and
conclusions of the work described. It should clearly state the paper's contribution to the
DiGRA community
\section*{FIGURES AND TABLES}
Place figures and tables at the top or bottom of the appropriate column or columns, on the
same page as the relevant text (see Figure 1). A figure or table may extend to a maximum
width of 14 cm (5.5 in.). If possible, the Figure should appear on the same page in which
it is first referenced.

Captions should be Times New Roman 11-point bold. They should be numbered (e.g.,
``Table 1'' or ``Figure 2''), centered and placed beneath the figure or table. Please note that
the words ``Figure'' and ``Table'' should be spelled out (e.g., ``Figure'' rather than ``Fig.'')
wherever they occur.  To do this in LaTeX use \textbackslash  caption*\{\textbf{Table 1}\} instead of  \textbackslash  caption\{\}, i.e. you will need to suppress the normal numbering and do it yourself.

Papers and notes may use color figures, which are included in the page limit. The paper
may be accompanied by various media files, such as videos, sound clips, and even
demonstrator games. However, the paper should stand on its own without such media, as
they may not be available to everyone who reads the paper.
\begin{figure}[h]
\centering
\begin{tabular}{| l | c | c | c |}
\hline
  & \textbf{Male} & \textbf{Female} & \textbf{Did not Respond} \\
  \hline
Digital &43 &36 &21\\
\hline
Non-Digital &42 &53 &5 \\
\hline
\end{tabular}
\caption*{\textbf{Table 1}: This is the title for my table, it is justified both left and right, and it is in 12
points. Try to centre the text of the table as shown. Table content may be set in smaller}
\end{figure}

\section*{ENDNOTES, BIBLIOGRAPHY AND LUDOGRAPHY}
\subsection*{Endnotes}
DiGRA recommends the use of endnotes\endnote{This is another example of an endnote.
} rather than footnotes. These should be placed
after the body text, but before the Bibliography section and numbered 1, 2, 3 etc.
Endnotes are also in 12 point Times New Roman.

\subsection*{Citations and References}
DiGRA uses a simplified version of the Chicago citation system. In running text and
endnotes, use \textbackslash cite\{key\} to generate a citation in parenthesis format (Author last names Publication-year).
To include a page or chapter number, use \textbackslash cite[page]\{key\}, producing
(Author last names Publication year, Page/Chapter). If the
authors' name is mentioned in running text, use \textbackslash textcite instead of \textbackslash cite.
This places the authors' name(s) in the text, and only the publication year in parenthesis.
\textcite[453]{ethics} may or may not think this is a good idea, but it doesn't matter since
this sentence is only included as an example. If a reference has more than two authors
\cite{schwartz.1995}, use the name of the first author ``et al.'' in the reference. 

Organize the bibliography alphabetically by last name of the first author. See the
bibliography towards the end of the template for example formatting of references.
\subsection*{Game references}
For games, set the game's name in italics with initial capitals. On first occurrence in the
text, include the developer and publication year in parenthesis, e.g. \textit{World of Warcraft}
\nocite{WoW}. There doesn't seem to be a clear way to cite a game in LaTeX so it is recommended to use the misc type in the bibliography (see the attached bib file for an example) and to do something like \textbackslash textit\{World of Warcraft\} \textbackslash nocite\{WoW\}.  Depending on the use of the game in the context of the article, you may
also choose to refer to the principal designer(s), creator(s), and so on. Follow this format
as closely as possible:
Developer. (Year). Title. [Platform, Version], Publisher, Release City/State and Country:
played day month, year.
See the Bibliography section towards the end of this template for examples.
\section*{LANGUAGE AND STYLE}
The written and spoken language of DiGRA is English. Spelling and punctuation may use
any dialect of English (e.g., British, Canadian, US, etc.) provided this is done
consistently. To ensure suitability for an international audience, please pay attention to
the following:
\begin{itemize}
\item Write in a straightforward style. Try to avoid long or complex sentence
structures.
\item Briefly define or explain all technical terms that may be unfamiliar to readers.
\item Explain all acronyms the first time they are used in your text – e.g., ``Alternate
Reality Game (ARG)''.
\item Explain local references (e.g., not everyone knows that a child in the first grade
of school in the US is 6-7 years old).

\item Explain ``insider'' comments. Ensure that your whole audience understands any
reference whose meaning you do not describe (and do not assume that everyone
has read a particular article).
\item Avoid or explain colloquial language and puns. Humor and irony are difficult to
translate.
\item Use unambiguous forms for culturally localized concepts, such as times, dates,
currencies and numbers (e.g., ``1-5- 97'' or ``5/1/97'' may mean 5 January or 1
May, and ``seven o'clock'' may mean 7:00 am or 19:00). For small currencies,
indicate equivalences in Euro or Dollar – e.g., ``Participants were paid 10,000 lire,
or roughly \$5.''
\item Be careful with the use of gender-specific pronouns (he, she) and other gendered
words (chairman, manpower, man-months). Use inclusive language that is
gender-neutral (e.g., she or he, they, s/he, chair, staff, staff-hours, person-years).
\item If possible, use the full (extended) alphabetic character set for names of persons,
institutions, and places (e.g., Grønbæk, Lafreniére, Sánchez, Universität,
Weißenbach, Züllighoven, Århus, etc.). Avoid using non-latin alphabets for
concepts and names. Make sure to include latin transcriptions if this is necessary.
\end{itemize}
\section*{PRODUCING AND TESTING PDF FILES}
We recommend that you produce a PDF version of your submission well before the final
deadline. Besides making sure that you are able to produce a PDF, you will need to
check that (a) the length of the file remains within the submission category's page limit, if
applicable, (b) the PDF file size is 4 megabytes or less, and (c) the file can be read and
printed using Adobe Acrobat Reader.

Note that most reviewers will use a North American/European version of Acrobat reader,
which cannot handle documents containing non-North American or non-European fonts
(e.g. Asian fonts). Please therefore do not use Asian fonts, and verify this by testing with
a North American/European Acrobat reader (freely available from Adobe). Something as
minor as including a space or punctuation character in a two-byte font can render a file
unreadable.
\section*{BLIND REVIEW}
Some DiGRA submission categories require blind review. To prepare your submission
for blind review, remove author and institutional identities in the title and header areas of
the paper. To preserve formatting, we recommend replacing identifying information with
generic values (e.g. Anonymous Author at Anonymous Institution). You may also need to
remove part or all of the Acknowledgments text. Further suppression of identity in the
body of the paper and references is left to the authors' discretion. For more details, see the
submission guidelines and checklist for your submission category.
\section*{CONCLUSION}
It is important that you write for the DiGRA audience. Please read previous years'
Proceedings (available from the DiGRA library, http://www.digra.org/dl) to understand
the writing style and conventions that successful authors have used. It is particularly
important that you state clearly what you have done, not merely what you plan to do, and
explain how your work is different from previously published work, i.e., what is the
unique contribution that your work makes to the field? Please consider what the reader 
will learn from your submission, and how they will find your work useful. If you write
with these questions in mind, your work is more likely to be successful, both in being
accepted into the conference, and in influencing the work of our field.
\section*{ACKNOWLEDGMENTS}
This template (originally for the 2016 FDG/DiGRA conference) was based on the Word template used originally in DiGRA 2011 conference. Some of the references cited in this paper are included for illustrative
purposes only. Thanks go to Adam Summerville, Mark Nelson, and Mirjam P. Eladahri for work on this template.
\section*{ENDNOTES}
\addvspace{-1\baselineskip} %%I can't find another way to reduce the size of the gap for the end notes
\theendnotes

\section*{BIBLIOGRAPHY}
\printbibliography
\end{document}
